\documentclass[10pt,onecolumn]{witseiepaper}
\usepackage{KJN}
\usepackage{graphicx}
\usepackage{url}
\usepackage{tikz}

\newcommand{\ttt}{\texttt}

\title{ELEN4002/ELEN4012 - Minutes}
\thanks{School of Electrical \& Information Engineering, University of the
Witwatersrand, Private Bag 3, 2050, Johannesburg, South Africa}

\begin{document}

\maketitle
\pagestyle{plain}
\setcounter{page}{1}

\section*{MEETING 1}
\subsection*{Attendees:}
Kayla-Jade Butkow, Kelvin da Silva, Prof. David Rubin
\subsection*{Agenda:} 
Obtain an overview of the requirements for the project outline specification

\subsection*{Minutes of meeting:}
Start date: 09/03/2018 \\
Start time: 13:00

Requirements for the document:
\begin{enumerate}
	\item Specifications of the project
	\item Milestones: Using Gantt Chart or timeline
	\item Preliminary budget and resources 
	\item Risks or Mitigation: What do you do if you can't get a piece of equipment (what do you fall back on)
\end{enumerate}

The minimum specifications for the project are as follows: 
\begin{itemize}
	\item There must be two hearing aids (one per ear)
	\item Each hearing aid must correct for the audiogram for that ear - To do this, you tweak the filter for each ear
	\item Directionality feature comes in mainly in processing - It is filtering of a signal based on direction (can possibly be done using cross correlation). This filtering changes the signal to noise ratio 
	\item Ideas for directionality: The user changes the direction using a potentiometer or the user always hears best in the direction they are facing
	\item You must be able to turn off directionality 
	\item Testing the device using humans is not specified in the brief of the project, but it can be done 
\end{itemize}

Testing the device:
\begin{itemize}
	\item Can use a signal generator to produce pure tones (single harmonic) and then examine the resulting signal after processing 
	\item To test directionality, the sound source can be moved and the SNR examined
	\item It is also necessary to produce signals in the presence of noise and test using them
\end{itemize}

Audiogram:
\begin{itemize}
	\item An audiogram is the response of each ear to different frequencies
	\item The magnitude in dB is with reference to some standard
	\item The layout of an audiogram is given in \figref{fig:audiogram}
\end{itemize}
\begin{figure}[h]
\centering
\begin{tikzpicture}[scale=3]
\draw[->] (0,0) -- (1,0) node[right] {$Frequency (Hz)$}; 
\draw[->] (0,0) -- (0,1) node[above] {$Magnitude (dB)$};
\end{tikzpicture}
\caption{Audiogram}
\label{fig:audiogram}
\end{figure}

Budget:
\begin{itemize}
	\item Is a huge constraint in the project
	\item It doesn't include any things you have in the house such as headphones 
\end{itemize}

Ethics Clearance:
\begin{itemize}
	\item For open day, you don't need ethics clearance
	\item We need to decide how many people we would like to use to test the device
	\item We need to specify who the people are (colleagues and lecturers)
	\item Provide a protocol for testing
\end{itemize}


\section*{MEETING 2}
\subsection*{Attendees:}
Kayla-Jade Butkow, Kelvin da Silva, Prof. David Rubin
\subsection*{Agenda:} 
Review ethics clearance application

\subsection*{Minutes of meeting:}
Start date: 03/04/2018 \\
Start time: 13:00

The ethics form was reviewed and appropriate changes were made

\section*{MEETING 3}
\subsection*{Attendees:}
Kayla-Jade Butkow, Kelvin da Silva, Prof. David Rubin
\subsection*{Agenda:} 
Obtain an overview of the requirements for the project planning document

\subsection*{Minutes of meeting:}
Start date: 20/06/2018 \\
Start time: 13:00

Project Planning Document:
\begin{itemize}
	\item Needs to Gantt chart which shows how the work will be split
	\item The document is a plan of the project assuming that everything is going to go exactly according to plan 
	\item Needs to detail the full project implementation - can possibly include circuit diagrams and flow charts or algorithms
\end{itemize}

Engineering notebook:
\begin{itemize}
	\item Use it as a diary - include dates and times for all entries
	\item Write down anything that you've done on that day - eg. sketch of a circuit, phone number of a supplier
	\item Put summaries of your minutes in the engineering notebook
\end{itemize}

The aim of the project is to develop a low cost hearing aid that is very simple to use. This means that the device only has a few buttons and is not fiddly. If we can achieve this, we can possibly work with Emoyo to develop a new low cost hearing aid solution.

The optimal directionality for the project would be that you are able to tune the direction using a dial, able to hear in the direction you are facing and able to turn of the directionality. This could be implemented using a combination of directional and non-directional microphones.

People to contact for the project:
\begin{itemize}
	\item James Braid - Contacts in speech and hearing if necessary
	\item Keegan Malan - Ask for use of Kuduwave to test device and to obtain audiograms
\end{itemize}

To Do:
\begin{itemize}
	\item Remind Prof. Rubin to email us audiograms
	\item Remind Prof. Rubin to enquire about a trial version of MATLAB for the project
\end{itemize}

Going forward, meetings are to be held once a week. Prof. Rubin will be away from the 29/06/2018 until 22/07/2018 and during this time, skype meetings will be held.

\section*{MEETING 4}
\subsection*{Attendees:}
Kayla-Jade Butkow, Kelvin da Silva, Dr. Derek Nitch
\subsection*{Agenda:} 
Consult regarding plans and intentions for the hearing aid directionality using array theory.

\subsection*{Minutes of meeting:}
Start date: 06/07/2018 \\
Start time: 10:00

Microphone Array:
\begin{itemize}
	\item The acoustic beam of the microphone array will not be able to be steered if the array only has two microphones.
	\item The array must have a minimum of four microphones to enable beam steering. Having more microphones allows for a more directive beam and improved resolution when steering.
	\item It was recommended that incoming sounds be constrained to the horizontal plane when performing simulations.
	\item Beam steering ability should be restricted to $180^{\circ}$ in front of the user.
	\item Inter-microphone spacing, \textit{d}, must be such that  the condition $d<\frac{c}{2 f_{max}}$ is met. \textit{c} is the propagation speed of sound ($343~m.s^{-1}$) and $f_{max}$ is the maximum frequency of interest.
\end{itemize}

Simulating array configurations:
\begin{itemize}
	\item Circular array with 4 or 6 microphones: There appears to be insignificant differences between each case. Azimuth polar pattern is acceptable but there is always two dominant beams in opposing directions.
	\item Semi-circle array with posteriorly placed plate: This configuration did not prove to be suitable as there were always more than two dominant beams. Posterior plate appears to be a good idea to incident sound waves from behind the array.
	\item Uniform linear array with posteriorly placed plate: This appears to be a promising configuration. There is a single main beam that remains dominant when steering it in a range of directions of $0^{\circ}-180^{\circ}$. The polar pattern obtained is acceptable when four microphones are used.
\end{itemize}

The minimum frequency range required to detect speech intelligibly must be determined to allow for optimal inter-microphone spacing. This will allow for a more consistent acoustic beam throughout all steering angles and a constant inter-microphone spacing.

To do:
\begin{itemize}
	\item Find the minimum frequency range required to detect speech intelligibly and use the upper bound of this range for the spacing of the microphones.
	\item Construct a weight table that indicates the phasing of each microphone required to focus the main acoustic beam in a specific direction. This should be done for the centre frequencies of each filter band.
	\item Obtain polar plots to verify the relevant weightings.
\end{itemize}

\section*{MEETING 5}
\subsection*{Attendees:}
Kayla-Jade Butkow, Kelvin da Silva, Prof. David Rubin
\subsection*{Agenda:} 
\begin{itemize}
	\item Discuss Arduino Due RAM limitations
	\item Discuss limiting frequencies of interest to speech frequency range and limiting directionality steering range
	\item Work station allocation
\end{itemize}

\subsection*{Minutes of meeting:}
Start date: 18/07/2018 \\
Start time: 10:00

Limiting frequency range and directionality steering range:
\begin{itemize}
	\item It is acceptable to limit the frequency range of interest too 0 - 8000~Hz and to limit directionality steering to the $180^{\circ}$ plane in front of the user.
\end{itemize}

RAM limitations:
\begin{itemize}
	\item It is expected that there will be a shortage of RAM if the designed filter bank is implemented digitally.
	\item Implementing the complete filter bank is seen to be important because it is an ANSI standard.
	\item It was concluded that the implementation of the hearing aid would require a dedicated IC. Additionally, the project time line is too short and the budget is too small to implement a completely modular design.
	\item It has therefore been decided that a complete, non-real time software design must be implemented. Thereafter a limited electronic/hardware version using simpler electronics will be implemented.
\end{itemize}

MATLAB software system design:
\begin{itemize}
	\item Full implementation of ANSI filter bank.
	\item Utilize pre-recorded audio files.
	\item Simulate detecting sound from different directions and show that directionality has been achieved.
	\item GUI for directionality: interactive dial to specify direction of desired hearing and polar plots of directivity.
	\item GUI for compensatory gain: Frequency spectrum of original and gained audio signals.
\end{itemize}

Hardware/electronic system design:
\begin{itemize}
	\item Pick three of the filter bands of the frequency bank and implement theme as analog filters.
	\item Gain relevant signals using the micro-controller.
	\item Focus on utilizing the processing capabilities of the micro-controller on directionality.
\end{itemize}

To do:
It has been recommended that the entire software solution must be completed first. Thereafter, the compensatory gain portion of the hardware version must be completed. The directionally feature of the hardware version should be implemented last.


\section*{MEETING 6}
\subsection*{Attendees:}
Kayla-Jade Butkow, Kelvin da Silva, Dr. Derek Nitch
\subsection*{Agenda:} 
Obtain advice on how to address the poor quality audio output when an audio sample is simulated from different directions in MATLAB.

\subsection*{Minutes of meeting:}
Start date: 24/07/2018 \\
Start time: 10:00

Simulated sound collection by array:
\begin{itemize}
	\item Discussed how the collection has already been implemented using the \textit{phased.WidebandCollector} MATLAB function.
	\item Check to see if there was any distortion of original audio file immediately after the microphone array. There was no distortion.
\end{itemize}

Filter bank:
\begin{itemize}
	\item Demonstrated the performance of the filter bank evaluating distortion of the audio before and after the filter bank. There was a clicking noise that was present in the audio.
	\item Clicking noise was improved by passing the entire audio stream through the filter bank at once compared to when passed through in frames.
\end{itemize}

Audio with applied weights passing through filter bank:
\begin{itemize}
	\item Through demonstration, the audio output is severely distorted even when the collected audio is multiplied by 1.
	\item It is found that each filter needs to be reset/released before passing consecutive frames through the filter.
	\item Weights were being applied to collected signals incorrectly as frequency domain weightings were being applied to time domain signals
	\item Weights (currently imaginary numbers) representing phase shifts have to be converted to time delays. It was suggested that by doing this and adding delayed signals of all filters bands appropriately, the distorted output audio would be improved.
\end{itemize}

Analog filter bank:
When we pick the filter bands, put them as far apart as possible. This will mean that the order of the filters does not have a large effect on their performance.

To do:
\begin{itemize}
	\item Convert microphone weights for all angles and centre frequencies to time delays.
	\item Implement time delays as array element shifts and add all arrays together appropriately.
	\item Vectorise all the for loops in order to speed up the execution time of the code
\end{itemize}

\section*{MEETING 7}
\subsection*{Attendees:}
Kayla-Jade Butkow, Kelvin da Silva, Prof. David Rubin
\subsection*{Agenda:} 
\begin{itemize}
	\item Weekly progress report
	\item Polar plot frequencies for the GUI
	\item How do we calculate the required gain for the compensatory amplification?
	\item How closely does the software have to match the hardware?
\end{itemize}

\subsection*{Minutes of meeting:}
Start date: 25/07/2018 \\
Start time: 12:00

Polar Plot:
Pick a few frequencies (5) and plot them to show the polar patterns at different frequencies

Compensatory gain:
\begin{itemize}
	\item The required gains are just the gains on the audiogram
	\item We have implemented the compensatory gain correctly
\end{itemize}

Software vs hardware:
\begin{itemize}
	\item We can use any number of microphones in software - the hardware and software number of microphones do not have to match
	\item We don't have to implement 3 of the software filters in hardware
	\item However, it is not ideal to use 3 bands that span the entire frequency range of speech as this will not accurately show how the compensatory amplification works
	\item It is fine to use three bands that are very far apart
	\item We should expand as much as we can in the software to create a full hearing aid simulation
\end{itemize}

Report:
\begin{itemize}
	\item It is essential that we talk about the cost and time limitations
	\item We must have a discussion about where the project needs to go to be implemented in real life (in the future recommendations section)
\end{itemize}

To do:
At the end of the project, remind Prof. Rubin to thank Kirsten at Optinum for the MATLAB licences.

\section*{MEETING 8}
\subsection*{Attendees:}
Kayla-Jade Butkow, Kelvin da Silva, Prof. David Rubin
\subsection*{Agenda:} 
\begin{itemize}
	\item Dynamic compression
	\item Title changes
	\item Planning report feedback
	\item Compensatory amplification for both ears vs one ear
\end{itemize}

\subsection*{Minutes of meeting:}
Start date: 01/08/2018 \\
Start time: 10:00

Planning report:
\begin{itemize}
	\item Report was good in terms of style and content
	\item Content is not really relevant since the re-scope
\end{itemize}

Title changes:
\begin{itemize}
	\item Option 1: "Towards the design of an adaptive hearing aid"
	\item Option 2: "An investigational study into the design of a low-cost adaptive hearing aid"
	\item Meaning of adaptive in this context: Adapts to the user's audiogram and to their environment (directional/omni-directional)
	\item Email Hugh Hunt for the title change procedure 
	\item Option 2 is the best option
\end{itemize}

Dynamic Compression:
\begin{itemize}
	\item Talk to Prof Hanrahan
\end{itemize}

Compensatory amplification:
\begin{itemize}
	\item If we have time, it would be ideal to apply separate amplification for both ears and then to output two analog signals
	\item If not, do one ear and say that it proves the concept that the amplification is working 
\end{itemize}

Advice from Prof Rubin:
\begin{itemize}
	\item For the audio amplifier, the inbuilt controllable gain might have less noise than a voltage divider
	\item For the GUI, we must make a recording of our voices
\end{itemize}

Polar plots:
\begin{itemize}
	\item It is not ideal to plot the polar plot on a rectangular axis
	\item Try to plot the line plot as a 2D scatter plot according to Equations \ref{eqn:pnew} and \ref{eqn:xy}. Then join the points together
	\item So that the calculations don't add extra delays into the system, pre-calculate all the points and store them into a 2D array. Then just pick the correct row/column to plot
\end{itemize}

\begin{equation}
P_{new}= \frac{P_{old} - min}{max - min} 
\label{eqn:pnew}
\end{equation}

\begin{equation}
(x,y)= (P_{new}\cos \theta, P_{new}\sin \theta)
\label{eqn:xy}
\end{equation}

\section*{MEETING 9}
\subsection*{Attendees:}
Kayla-Jade Butkow, Kelvin da Silva, Prof. David Rubin
\subsection*{Agenda:} 
\begin{itemize}
	\item Weekly progress report
	\item Arduino sampling frequency
	\item Project title changes
\end{itemize}

\subsection*{Minutes of meeting:}
Start date: 08/08/2018 \\
Start time: 10:00

Arduino sampling frequency:
\begin{itemize}
	\item Monotone sounds that are being sampled are distorted due to a low sampling frequency. 
	\item The sampling frequency gets lower as more lines of code get added.
	\item The obvious solution is to remove one frequency band per microphone to keep sampling frequency as high as possible (Plan B).
	\item Another proposed solution is the use of two Arduinos that serially communicate with one another.
	\item Have each Arduino deal with separate frequency bands and synchronize them to avoid significant undesirable delays.
	\item MAke use of an external ADC chip for reading potentiometer values to ensure that sampling frequency of Arduino is maximised. Do not waste a channel on simply reading a potentiometer voltage
\end{itemize}

Advice on report content:
\begin{itemize}
	\item State that a major limitation of the hardware design is the use of the Arduino Due.
	\item State the reasoning for use of two Arduinos: Increasing sampling frequency and preventing interference between each frequency band.
\end{itemize}

Title changes:
\begin{itemize}
	\item Action required for title changes only involves sending the new title to Mr Hugh Hunt.
\end{itemize}

Arduino interrupts:
\begin{itemize}
	\item To determine the sampling frequency, it is recommended that a timer is started at the end of program set-up and ended once the sample counter is 50 (end of interrupt).
\end{itemize}

To do:
\begin{itemize}
	\item Investigate the use of timer-based interrupts to improve sampling frequency.
	\item Only move onto the approach using two Arduinos once the system functions as expected when dealing with a single frequency band.
\end{itemize}

\section*{MEETING 10}
\subsection*{Attendees:}
Kayla-Jade Butkow, Kelvin da Silva, Prof. David Rubin
\subsection*{Agenda:} 
\begin{itemize}
	\item Weekly progress report
	\item Directionality feature issues
	\item Poster details
	\item Open-Day demonstration
\end{itemize}

\subsection*{Minutes of meeting:}
Start date: 15/08/2018 \\
Start time: 10:00

Sampling frequency:
\begin{itemize}
	\item The issue was rectified by making use of a timer based interrupt.
\end{itemize}

Directionality:
\begin{itemize}
	\item The converted MATLAB code for directionality only functions correctly for a single direction ($90^{\circ}$).
	\item If directionality is cannot be achieved in other directions, reasoning must be provided in the report.
	\item It is suggested that a logic analyser should be used to monitor the behaviour of each channel of the Arduino for the purpose of determining the reason why the program malfunctions.
\end{itemize}

Open-Day:
\begin{itemize}
	\item Hardware demonstration: Connect an oscilloscope to the output of the device and move a sound source around to demonstrate a change in signal magnitude when in directional mode.
	\item Hardware demonstration: A similar procedure is to be taken in omnidirectional mode.
	\item Software demonstration: Make use of the developed GUI.
	\item Permission granted for use of Professor Rubin's digital oscilloscope for testing and on Open-Day.
\end{itemize}

Testing of software:
\begin{itemize}
	\item Test the software solution in exactly the same way that the hardware is tested.
	\item Prove that the gain applied to a signal (in accordance with a given audiogram) manipulates the signal appropriately.
	\item Test for a range of frequencies and gain.
	\item Test directionality by obtaining polar plots when the acoustic beam is facing different directions and show the effect on magnitude of the applied signal.
\end{itemize}

Poster:
\begin{itemize}
	\item The poster must mainly consist of pictures and as few words as possible.
	\item Make use of bullet points, large font and include a high level block diagram of the developed system.
	\item An abstract should be included whereby the reader is informed about what the project entails.
	\item Polar plots and frequency spectrum plots will be useful aids for a results section.
	\item Present the error in directionality as well as compensatory gain.
\end{itemize}

To do:
\begin{itemize}
	\item Look at technical conference posters for guidance.
	\item Design a casing for the hardware solution to be housed in.
\end{itemize}










\end{document}